\documentclass{article}
\usepackage{CJKutf8}
\usepackage{natbib}
\bibliographystyle{abbrvnat}

\newcommand{\hezehui}{
  \begin{CJK*}{UTF8}{gbsn}
    何泽慧
\end{CJK*}
}

\begin{document}

\section{The Question}

\begin{quote}
The reciprocal of what quantity was estimated to be .9 higher than the
three-digit integer value popularly associated with it in 1916 experiments
conducted by a Brandenburg-based researcher?  That researcher also hosted Ho Zah-wei (\hezehui{}) in his
home; she would later go on to lead the neutron flux team of the Chinese nuclear program.
\end{quote}

The answer is the \underline{fine structure constant}, $\alpha \approx \frac{1}{137}$.

\section{Explaining the Answer}

He Zehui was one of the leading minds of the Chinese nuclear program,
but she was educated in Germany; while there, she stayed at the home
of Friedrich Paschen~\citep{cern-11}.  Paschen was notably one of the
foremost investigators of spectral lines, which led to his estimation
of the reciprocal of the fine structure constant as 137.9, confirming
the theory of Sommerfeld.

However, the fine structure constant is typically thought of as
$\frac{1}{137}$, whose integer value was described as Feynman as being
``written by the hand of God''~\citep{feynman-98}.

\section{Why it's Interesting}

\paragraph{Why ask about this topic?}
The fine structure constant is worthy to ask about because it is one
of the few constants in physics that doesn't need to be described in
terms of units, in contrast to constants like the speed of light which
require things like meters and seconds to be defined.  In other words,
you could communicate the fine structure constant more easily to an
alien than other fundamental physics constants.  (One could argue that
$\pi$ our prime numbers would also fit the bill.)

\paragraph{Why are these facts interesting?}
The question asks about the specific value of the constant, which is
still being measured recently~\citep{kleppner-06}.  It also connects
Friedrich Paschen who made the \emph{first} measurement of this
constant to He Zehui, who was critical to the Chinese nuclear
program~\citep{lewis-91}.

\paragraph{Why is this gettable for a knowledgable human?}
The hint about the constant being nearly an integer is often asked
about in trivia questions:
\begin{itemize}
  \item This constant, used to measure some electromagnetic interactions, equals Coulomb’s
constant times elementary charge squared divided by the quantity h-bar times c. It is
approximately 1 over 137.
\item The fine structure constant is dimensionless, and has the same value in any unit system. Its value is usually
  approximated as 1 over this integer.
  \item For 10 points, name this coupling
constant of electromagnetic interactions, which is equal to approximately one over 137 and is symbolized
alpha.
\end{itemize}
And some questions even ask for the integer value.  So this is
difficult for a good trivia player, but not impossible.

\section{Why it's hard}

A simpler version of this question would be: Friedrich Paschen
estimated that the reciprocal of what constant proposed by Sommerfeld
was 137.9 (not an integer!) in a 1916 experiment?

This question is substantially easier to answer because the
fine-structure constant is popularly associated with the integer 137
(one of the few constants not to be a long decimal).

Moreover, it mentions Sommerfeld, a relatively well-known German
physicist, who proposed the fine-structure
constant~\citep{sommerfeld-16}.

We make it harder by adding a multi-hop step~\citep{yang-18} to connect
to Paschen and then forcing the system to do numerical reasoning to
get the clue that this value is nearly an integer~\citep{dua-19}.

We also make it harder by giving a toponym of Potsdam (Brandenburg)
and giving an alternate transliteration of \hezehui{} rather than the more
common pinyin.

The additional clue about He Zehui can help point a human in the right
direction but might distract the computer.

\section{What do QA systems answer?}

\begin{itemize}
  \item {\bf ChatGPT}: Unfortunately, the given information is not sufficient to determine the answer to this question.
  \item {\bf You.com}: The quantity being referred to in the question
    is the proton-electron mass ratio. In 1916, a researcher named
    Theodore Lyman conducted experiments to estimate the value of this
    ratio, and he found that the reciprocal of the ratio was about 0.9
    higher than the three-digit integer value that was popularly
    associated with it at the time.  The researcher who hosted Ho
    Zah-wei in his home was James Franck, who was a friend and
    colleague of Lyman's.
  \end{itemize}


\bibliography{references}
  
  \end{document}